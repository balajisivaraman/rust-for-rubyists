\section{Rust Programming Language}
\label{sec:rust-language}

\begin{frame}[fragile]
  \frametitle{Can you make sense of this?}
    \begin{minted}{rust}
      pub fn main() {
        let v = vec!(1,2,3);
        let numbers: Vec<i32> = v.iter().map(|n| n * n).collect();
        println!("{:?}", numbers);
      }
    \end{minted}
\end{frame}

\begin{frame}
  \frametitle{Rust Benefits}
  \begin{itemize}
  \item High-Level Syntax (similar to Ruby or Java)
  \item Low-Level Performance (similar to C/C++)
  \end{itemize}
\end{frame}

\begin{frame}[fragile]
  \frametitle{Rust {\DejaSans ❤} TDD}
    \begin{minted}{rust}
      pub fn add_one(a: u32) -> u32 {
        a + 1
      }

      #[test]
      fn test_add() {
        let result = add_one(1);
        assert_eq!(result, 2);
      }
    \end{minted}
    Compile and Run
    \begin{minted}{bash}
      $ cargo --test sample.rs
      $ ./sample
    \end{minted}
\end{frame}

\begin{frame}
  \frametitle{Key Language Features}
  \begin{itemize}
  \item Functional Language Features I like in Rust
    \begin{itemize}
    \item Pattern Matching
    \item ENums similar to Algebraic Data Types
    \item Lazy Iterators
    \item Functions as first class values
    \item Error Handling Primitives using Result ADT
    \end{itemize}
  \item Rust Lang Features I Like
    \begin{itemize}
    \item \textbf{Ownership and Borrowing}
    \item Lifetimes
    \item Unit Testing primitives as part of the core language
    \item Concurrency Primitives --- Threads, Channels, Atomic Values etc.
    \end{itemize}
  \end{itemize}
\end{frame}
