\documentclass[bigger]{beamer}
\usepackage[utf8]{inputenc}
\usepackage{fixltx2e}
\usepackage{graphicx}
\usepackage{grffile}
\usepackage{longtable}
\usepackage{wrapfig}
\usepackage{rotating}
\usepackage[normalem]{ulem}
\usepackage{amsmath}
\usepackage{textcomp}
\usepackage{amssymb}
\usepackage{capt-of}
\usepackage{hyperref}
\usepackage[backend=biber,style=numeric]{biblatex}
\bibliography{rust-for-rubyists.bib}
\usepackage{fontspec}
\newfontfamily\DejaSans{DejaVu Sans}
\usepackage[outputdir=output]{minted}
\usepackage{tikz}
\usetikzlibrary{tikzmark}
\setmonofont{PragmataPro}
\setminted{autogobble=true,fontsize=\footnotesize}

\setbeamersize{text margin left=0.6cm,text margin right=0.6cm}
\setbeamertemplate{bibliography item}{\insertbiblabel}

\usetheme{Copenhagen}

\author{Balaji Sivaraman (\href{https://twitter.com/balajisivaraman}{@balajisivaraman})}
\date{March 24, 2018}
\title{Rust for Rubyists}
\institute{ThoughtWorks}
\hypersetup{
  pdfauthor={Balaji Sivaraman},
  pdftitle={Rust for Rubyists},
  pdfkeywords={rust-lang,systems-programming,memory-management,lifetimes},
  pdfsubject={A gentle introduction to Rust Lang for Rubyists, highlighting salient features of the former},
  pdflang={English}}

\begin{document}

\maketitle

\begin{frame}{About Me}
  \begin{itemize}
  \item Primarily worked on Java/Spring/ROR stack in ThoughtWorks, writing microservices
  \item Pure functional programming advocate in languages like Scala/Haskell/Purescript
  \item Bitten by the Rust bug last year after reading a post on how \
    it enabled Firefox's superior performance
  \item Currently on the way to transitioning from an applications \
    developer to a systems programmer, thanks primarily to Rust
  \end{itemize}
\end{frame}

\section{Introduction}
\label{sec:introduction}

\begin{frame}
  \frametitle{Agenda}
  \begin{itemize}
  \item Introduction
  \item Ground Rules
  \item Ruby Recap
  \item Systems Programming
  \item The Rust Language
  \item Why Rust --- The Good Stuff
  \item Questions?
  \end{itemize}
\end{frame}

\begin{frame}
  \frametitle{Ground Rules}
  What this talk is about?
  \begin{itemize}
  \item How Rust benefits people migrating from web to systems programming?
  \item What makes the Rust language unique?
  \item What benefit could I, as a Ruby programmer, get from it?
  \end{itemize}
  What this talk is not about?
  \begin{itemize}
  \item Convince you to stop using Ruby and start using Rust
  \end{itemize}
\end{frame}

\section{Ruby Recap}
\label{sec:ruby-recap}

\begin{frame}
  \frametitle{What's to love?}
  \begin{itemize}
  \item Minimally Functional
  \item Syntax is lovely
  \item Low bootstrapping cost
  \item Bundler and Gems are great (But\ldots)
  \item Great for quickfire checks on IRB or small scripts
  \item Strongly typed (even if it is dynamic)
  \item Benefits TDD practitioners more than any other language
  \end{itemize}
\end{frame}

\begin{frame}
  \frametitle{What's not to love?}
  \begin{itemize}
  \item Dynamically Typed
  \item Performance (Invariably?)
  \item Concurrency (Really hard to get right)
  \item Bundling (Anyone faced Nokogiri issues?)
  \item Packaging and Deployment (Size of bundled package?)
  \end{itemize}
\end{frame}

\section{Systems Programming}
\label{sec:systems-programming}

\begin{frame}
  \frametitle{Systems Programming}
  \begin{itemize}
  \item What is Systems Programming?
  \item Why is it different from web/application programming?
  \end{itemize}
\end{frame}

\begin{frame}
  \frametitle{What is Systems Programming?}
  \begin{block}{From O'Reilly's Programming Rust~\cite{ProgRustPreface1}:}
    \begin{quotation}
      Systems programming is \textbf{resource-constrained}
      programming. It is programming when every byte and every CPU cycle
      counts.
    \end{quotation}
  \end{block}
\end{frame}

\begin{frame}
  \frametitle{What does that mean?}
  \begin{itemize}
  \item Programmer can almost never trade-off on performance
  \item No GC
  \item Minimal/No Runtime
  \item Zero-Cost Abstractions~\cite{Stroustrup}
  \end{itemize}
\end{frame}

\section{Rust Programming Language}
\label{sec:rust-language}

\begin{frame}
  \centerline{
    \huge{The Rust Programming Language}
  }
\end{frame}

\begin{frame}[fragile]
  \frametitle{Can you make sense of this?}
    \begin{minted}{rust}
      pub fn main() {
        let v = vec!(1,2,3);
        let numbers: Vec<i32> = v.iter().map(|n| n * n).collect();
        println!("{:?}", numbers);
      }
    \end{minted}
\end{frame}

\begin{frame}
  \frametitle{Rust Benefits}
  \begin{itemize}
  \item High-Level Syntax (similar to Ruby or Java)
  \item Low-Level Performance (similar to C/C++)
  \end{itemize}
\end{frame}

\begin{frame}[fragile]
  \frametitle{Rust {\DejaSans ❤} TDD}
    \begin{minted}{rust}
      pub fn add_one(a: u32) -> u32 {
        a + 1
      }

      #[test]
      fn test_add() {
        let result = add_one(1);
        assert_eq!(result, 2);
      }
    \end{minted}
\end{frame}

\begin{frame}[fragile]
  \frametitle{Running tests is straightforward}
    \begin{minted}{bash}
      $ cargo --test sample.rs
      $ ./sample
      running 1 test
      test test_add ... ok
      test result: ok. 1 passed; 0 failed; 0 ignored;
      0 measured; 0 filtered out
    \end{minted}
\end{frame}

\begin{frame}[fragile]
  \frametitle{Mimic OO}
  Let's define a simple Trait! \break{}
  \begin{minted}{rust}
    trait Animal {
      fn walk(&self);
    }
  \end{minted}
\end{frame}

\begin{frame}[fragile]
  \frametitle{Mimic OO}
  Let's implement it! \break{}
  \begin{minted}[fontsize=\scriptsize]{rust}
    struct Cat {
      name: String
    }

    impl Animal for Cat {
      fn walk(&self) {
        println!("{} walks like a cat", self.name);
      }
    }
  \end{minted}
\end{frame}

\begin{frame}[fragile]
  \frametitle{Mimic OO}
  Let's implement it again! \break{}
  \begin{minted}[fontsize=\scriptsize]{rust}
    struct Dog {
      name: String
    }

    impl Animal for Dog {
      fn walk(&self) {
        println!("{} walks like a dog", self.name);
      }
    }
  \end{minted}
\end{frame}

\begin{frame}[fragile]
  \frametitle{Mimic OO}
  What is the output? \break{}
  \begin{minted}{rust}
    fn main() {
      let d = Dog { name: String::from("Snuggles") };
      let c = Cat { name: String::from("Pinkie Pie") };
      d.walk();
      c.walk();
    }
  \end{minted}
\end{frame}

\begin{frame}
  \frametitle{Key Language Features}
  \begin{itemize}
  \item Functional Language Features I like in Rust
    \begin{itemize}
    \item Pattern Matching
    \item ENums similar to Algebraic Data Types
    \item Lazy Iterators
    \item Functions as first class values
    \item Error Handling Primitives using Result ADT
    \end{itemize}
  \item Rust Lang Features I Like
    \begin{itemize}
    \item \textbf{Ownership and Borrowing}
    \item Lifetimes
    \item Unit Testing primitives as part of the core language
    \item Concurrency Primitives --- Threads, Channels, Atomic Values etc.
    \end{itemize}
  \end{itemize}
\end{frame}


\begin{frame}
  \centerline{
    \huge{Questions?}
  }
\end{frame}

\begin{frame}
  \frametitle{References}
  \renewcommand*{\bibfont}{\scriptsize}
  \printbibliography{}
\end{frame}

\begin{frame}
  \centerline{
    \huge{Thank you!}
  }
  \centerline{
    \footnotesize{Slides source available at: \href{https://github.com/balajisivaraman/rust-for-rubyists}{https://github.com/balajisivaraman/rust-for-rubyists}}
  }
\end{frame}
\end{document}
