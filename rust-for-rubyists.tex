\documentclass[bigger]{beamer}
\usepackage[utf8]{inputenc}
\usepackage{fixltx2e}
\usepackage{graphicx}
\usepackage{grffile}
\usepackage{longtable}
\usepackage{wrapfig}
\usepackage{rotating}
\usepackage[normalem]{ulem}
\usepackage{amsmath}
\usepackage{textcomp}
\usepackage{amssymb}
\usepackage{capt-of}
\usepackage{hyperref}
\usepackage[backend=biber,style=numeric]{biblatex}
\bibliography{rust-in-peace.bib}
\usepackage{fontspec}
\usepackage[outputdir=output]{minted}
\usepackage{tikz}
\usetikzlibrary{tikzmark}
\setmonofont{PragmataPro}
\setminted{autogobble=true,fontsize=\footnotesize}

\setbeamersize{text margin left=0.6cm,text margin right=0.6cm}
\setbeamertemplate{bibliography item}{\insertbiblabel}

\usetheme{Copenhagen}

\author{Balaji Sivaraman (\href{https://twitter.com/balajisivaraman}{@balajisivaraman})}
\date{March 24, 2018}
\title{Rust for Rubyists}
\institute{ThoughtWorks}
\hypersetup{
  pdfauthor={Balaji Sivaraman},
  pdftitle={Rust for Rubyists},
  pdfkeywords={rust-lang,systems-programming,memory-management,lifetimes},
  pdfsubject={A gentle introduction to Rust Lang for Rubyists, highlighting salient features of the former},
  pdflang={English}}

\begin{document}

\maketitle

\begin{frame}{About Me}
  \begin{itemize}
  \item Primarily worked on Java/Spring/ROR stack in ThoughtWorks, writing microservices
  \item Pure functional programming advocate in languages like Scala/Haskell/Purescript
  \item Bitten by the Rust bug last year after reading a post on how \
    it enabled Firefox's superior performance
  \item Currently on the way to transitioning from an applications \
    developer to a systems programmer, thanks primarily to Rust
  \end{itemize}
\end{frame}

\section{Introduction}
\label{sec:introduction}

\begin{frame}
  \frametitle{Agenda}
  \begin{itemize}
  \item Introduction
  \item Ground Rules
  \item Ruby Recap
  \item Systems Programming
  \item The Rust Language
  \item Why Rust --- The Good Stuff
  \item Questions?
  \end{itemize}
\end{frame}

\begin{frame}
  \frametitle{Ground Rules}
  What this talk is about?
  \begin{itemize}
  \item How Rust benefits people migrating from web to systems programming?
  \item What makes the Rust language unique?
  \item What benefit could I, as a Ruby programmer, get from it?
  \end{itemize}
  What this talk is not about?
  \begin{itemize}
  \item Convince you to stop using Ruby and start using Rust
  \end{itemize}
\end{frame}

\section{Ruby Recap}
\label{sec:ruby-recap}

\begin{frame}
  \frametitle{What's to love?}
  \begin{itemize}
  \item Minimally Functional
  \item Syntax is lovely
  \item Low bootstrapping cost
  \item Bundler and Gems are great (But\ldots)
  \item Great for quickfire checks on IRB or small scripts
  \item Strongly typed (even if it is dynamic)
  \item Benefits TDD practitioners more than any other language
  \end{itemize}
\end{frame}

\begin{frame}
  \frametitle{What's not to love?}
  \begin{itemize}
  \item Dynamically Typed
  \item Performance (Invariably?)
  \item Concurrency (Really hard to get right)
  \item Bundling (Anyone faced Nokogiri issues?)
  \item Packaging and Deployment (Size of bundled package?)
  \end{itemize}
\end{frame}


\begin{frame}
  \centerline{
    \huge{Questions?}
  }
\end{frame}

\begin{frame}
  \frametitle{References}
  \renewcommand*{\bibfont}{\scriptsize}
  \printbibliography{}
\end{frame}

\begin{frame}
  \centerline{
    \huge{Thank you!}
  }
  \centerline{
    \footnotesize{Slides source available at: \href{https://github.com/balajisivaraman/rust-for-rubyists}{https://github.com/balajisivaraman/rust-for-rubyists}}
  }
\end{frame}
\end{document}
